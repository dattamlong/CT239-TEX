\phantomsection
\setsection{Chương 1: Tổng quan}
\setcounter{section}{1}

% \section*{\centering CHƯƠNG 1: TỔNG QUAN}
\phantomsection
\subsection{Mô tả về bài toán}
Bài toán phân công công việc cho nhân viên (Job Assignment Problem) là một bài toán tối ưu hóa quan trọng trong quản lý nguồn lực và tổ chức công việc. Trong bài toán này, giả sử có N công nhân và N công việc, và mỗi công nhân có thể được phân công thực hiện bất kỳ công việc nào. Chi phí cho việc phân công công việc có thể thay đổi tùy theo sự kết hợp giữa công nhân và công việc. Mục tiêu là phân công tất cả các công việc sao cho mỗi công nhân được phân công đúng một công việc và mỗi công việc được thực hiện bởi đúng một công nhân, đồng thời tổng chi phí của sự phân công là nhỏ nhất.

Để giải quyết bài toán này, có thể áp dụng một số thuật toán tối ưu khác nhau, bao gồm thuật toán tham ăn, thuật toán nhánh và cận, và thuật toán quy hoạch động. Thuật toán tham ăn, mặc dù đơn giản và dễ triển khai, có thể không luôn tìm ra giải pháp tối ưu nhất do đặc thù của các lựa chọn cục bộ. Thuật toán nhánh và cận có khả năng cung cấp giải pháp tối ưu bằng cách chia nhỏ bài toán thành các phần con và cắt bỏ các giải pháp không khả thi, nhưng có thể tốn kém về mặt tính toán. Thuật toán quy hoạch động, với khả năng lưu trữ các kết quả trung gian, có thể tối ưu hóa quy trình tính toán và cung cấp giải pháp hiệu quả hơn cho các bài toán lớn hơn.

\phantomsection
\subsection{Mục tiêu của đề tải}
Đề tài này nhầm mục tiêu phát triển và áp dụng các phương pháp tối ưu hóa nhằm cải thiện hiệu quả phân công công việc cho nhân viên trong tổ chức. Cụ thể, mục tiêu là xây dựng một hệ thống phân công công việc cho nhiên viên sao cho mỗi công việc được giao cho đúng một nhân viên và mỗi nhân viên chỉ thực hiện một công việc, đồng thời tối thiểu hóa tổng chi phí liên quan đến phân công này. Để đạt được mục tiêu này, đề tài sẽ nghiên cứu và áp dụng các thuật toán tối ưu như thuật toán tham ăn, thuật toán nhánh và cận, và thuật toán quy hoạch động. Qua việc áp dụng các thuật toán này, đề tài nhằm mục đích cải thiện quy trình phân công công việc, đảm bảo sự phân bổ tài nguyên hợp lý, và nâng cao hiệu quả công việc, từ đó đạt được sự cân bằng tốt nhất giữa chi phí và hiệu suất công việc.

\phantomsection
\subsection{Hướng giải quyết và kế hoạch thực hiện}